\chapter{传统共识算法概述}
\label{cha:Simple}

\section{最简单的共识算法}

我们使用有向图的邻接矩阵A来描述其通信拓扑:

\begin{equation}
    \mathbf{A}=\left[a_{i j}\right]_{N \times N}
\end{equation}

$a_{\mathrm{ij}}$为A中的元素:

\begin{equation}
    a_{\mathrm{ij}}=\left\{\begin{array}{ll}
    {1} & {\text { if } j \in N_{i}} \\
    {0} & {\text { Otherwise }}
    \end{array}\right.
\end{equation}

其中N为与i节点相连通的节点集合。

\begin{equation}
    x_{\mathrm{i}}(t)=x_{\mathrm{i}}(t)+u_{\mathrm{i}}(t) \quad i=1,2,3,4 \ldots \ldots, n
\end{equation}

$u_{\mathrm{i}}(t)$为$x_{\mathrm{i}}(t)$的控制变量,服从:

\begin{equation}
    u_{i}(t)=\frac{1}{N_{\mathrm{i}}}\sum_{j=1}^{n} a_{i j}(t)\left[x_{j}(t)-x_{i}(t)\right]
\end{equation}

其中$N_{\mathrm{i}}$为与i节点相连通的节点数。

上述算法的物理含义即:

在第i次迭代过程中,同时对和每个节点相连通的所有节点取平均并赋值给此节点\cite{8706900}。