\chapter{基于马尔可夫链和Perron–Frobenius算法用于分布式电力系统经济调度的过渡矩阵Q优化}
\label{cha:Algorithm}

本章基于马尔可夫链和Perron–Frobenius算法阐述一种新的用于分布式电力系统经济调度的过渡矩阵Q优化方法

\section{概述}


\section{分布式经济调度模型}

\subsection{物理模型}

现阶段研究的物理模型公式如式~\ref{eq:phymodel},其中,N为机组数量,$p_{i}$为机组出力,$W_{i}\left(p_{i}\right)$为机组出力成本函数,$\lambda_{i}\left(p_{i}\right)$为边际成本函数,$\alpha_{i}$ $\beta_{i}$ $\gamma_{i}$ 均为机组出力成本函数中的参数,其中$\alpha_{i}$为机组的假想最低成本,$\beta_{i}$为边际成本关于机组出力边际增长率的倒数,可称为“出力-价格灵敏度”,$\gamma_{i}$为线性外推得到的零边际成本下的假想机组出力。$L_{i}$为节点i处的负荷。$\underline{p}_{i}$和$\bar{p}_{i}$分别是机组出力上限与下限。

\begin{equation}
    \begin{aligned}
    &\min W(\mathbf{P})=\sum_{i=1}^{N} W_{i}\left(p_{i}\right)\\
    &\sum_{i=1}^{N} p_{i}=\sum_{i=1}^{N} L_{i}\\
    &\underline{p}_{i} \leq p_{i} \leq \bar{p}_{i}, \forall i\\
    &W_{i}\left(p_{i}\right)=\frac{\left(p_{i}-\alpha_{i}\right)^{2}}{2 \beta_{i}}+\gamma_{i}, i=1,2, \ldots, N\\
    &\lambda_{i}\left(p_{i}\right)=\frac{\partial W_{i}\left(p_{i}\right)}{\partial p_{i}}=\frac{1}{\beta_{i}}\left(p_{i}-\alpha_{i}\right), i=1,2, \ldots, N\\
    &p_{i}=\beta_{i} \lambda_{i}+\alpha_{i}
    \end{aligned}
    \label{eq:phymodel} 
\end{equation}

在边际成本函数的基础上,目标函数等价于求价格共识$\lambda$,其含义是:

\begin{itemize}
    \item 各节点处都有边际成本$\lambda_{i}$,必须保证全部$\lambda_{i}$相等或足够接近。
    \item 在$\lambda_{i}$下,由式~\ref{eq:phymodel}-6决定的机组出力$p_{i}$必须满足全部约束条件。
\end{itemize}


