\chapter{移动平台设计和制造}
\label{cha:Platform}

\section{平台设计}
形式:

每个小车为一个数据点,在1-3维坐标系内动态运动(如果和ShapeBots\cite{suzuki2019shapebots}一样具备升降平台或者和SwarmOS一样具备可变色LED点阵则可以进行三维变量显示)

也可以每个小车作为一个通信节点(在电力市场的应用场景下是一个机组节点),用屏幕/位置/RGB颜色/高度直观的显示迭代的过程。

即插即用的实现:放入/取出小车,新的迭代随即开始。

\subsection{仿真}
https://ryosuzuki.github.io/shapebots-simulator/

src https://github.com/ryosuzuki/shapebots-simulator

\subsection{定位方法}

\subsection{底盘}

\subsection{主控板}

\subsection{扩展接口}

\subsection{实物可视化界面}

\subsection{UI}