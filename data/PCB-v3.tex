\chapter{PCB 3.0 改进设计和测试}
\label{cha:PCB-v3}

\section{待改进问题}

三路驱动丝印编号

电池插头+-反向,以丝印标注

芯片周围留出clearance,方便拖焊。摆放可以考虑45度角摆放,方便密集的布线。

最好改成单面有贴片元件的形式,方便SMT或者制作钢网涂锡膏。

选用没有ThermalVias的芯片封装,否则容易焊锡粘到了ThermalVias上引起不平整。(但是没有不含ThermalPad的封装)

留出STEP和DIR测试/备用引脚,以便板上有一两片DRV8825无法正常使用可以外接模块。

12V和5V供电太细了,要注意供电可靠性。

扩大PCB面积,从100mm到120mm直径圆内接正六边形。

Reset按钮封装不对。Reset电路中R22/D1,RESET BUTTON/C13不是必须的,可以画,可以不焊。

!!!发现CH340C的TX接到了Mega的TX上,RX接到了Mega的RX上,所以烧录Arduino时出现错误:

\begin{tcolorbox}
    avrdude: stk500v2\_ReceiveMessage(): timeout \\
    avrdude: stk500v2\_getsync(): timeout communicating with programmer
\end{tcolorbox}
