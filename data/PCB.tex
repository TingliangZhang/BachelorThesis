\chapter{主板及扩展板PCB设计}
\label{cha:PCB}

主板PCB设计如图~\ref{fig:CorePCB}:

\begin{figure}[htbp]
    \centering
    \includegraphics[width=\columnwidth]{ArduinoMega2560-Core-White-Crop.pdf}
    \caption{Mega 2560主板设计}
    \label{fig:CorePCB}
\end{figure}

\section{主控ATmega2560-16AU}

主控芯片采用ARDUINO MEGA 2560 REV3\cite{arduino_mega-2560-r3}上使用的ATmega2560-16U芯片\footnote{\href{http://www.atmel.com/Images/Atmel-2549-8-bit-AVR-Microcontroller-ATmega640-1280-1281-2560-2561_datasheet.pdf}{ATmega2560 Datasheet}},并参考MEGA 2560的芯片外设和烧写器设计。引脚映射图见附录~\ref{sec:Pin2560}。

\begin{figure}[htbp]
    \centering
    \includegraphics[width=0.7\textwidth]{PinMap2560big_Rev2.png}
    \caption{Mega 2560 PIN diagram}
    \label{fig:PinMap2560}
\end{figure}

ATmega2560是高性能,低功耗基于Microchip 8位AVR RISC的微控制器(MCU),256KB ISP闪存,8KB SRAM,4KB EEPROM,86个GPIO,32个通用工作寄存器,实时计数器,六个具有比较模式的灵活定时器/计数器,PWM,4个USART,面向字节的2线串行接口,16通道10位A/D转换器以及用于片上调试的JTAG接口。该器件在16 MHz时可达到16 MIPS的吞吐量,并在4.5至5.5伏之间工作。通过在单个时钟周期内执行指令,该设备可实现接近1MIPS/MHz的吞吐量,从而平衡了功耗和处理速度。

\begin{figure}[htbp]
    \centering
    \includegraphics[]{Mega2560-ATmega2560-16AU.pdf}
    \caption{Mega 2560 MCU 原理图}
    \label{fig:Mega2560-ATmega2560-16AU}
\end{figure}

\section{供电和稳压电路}

\begin{figure}[htbp]
    \centering
    \includegraphics[]{Mega2560-Power.pdf}
    \caption{供电和稳压电路}
    \label{fig:Mega2560-Power}
\end{figure}

如图~\ref{fig:Mega2560-Power},外部锂电通过PWRIN接口输入7-15v直流,两级滤波,使用小电容100nF滤除高频干扰,大电容47uF消除低频干扰,M7为二极管,防止错误输入负电压。低压差线性稳压(LDO)降压芯片LD1117S50CTR完成电平转换,输出5V直流,在输出端也加入了一个47uF的滤波电容,滤除输出5V中的低频谐波。

AVCC是端口F和A/D转换器的电源电压引脚。即使不使用ADC,它也应从外部连接到VCC。如果使用ADC,则应通过低通滤波器将其连接到VCC。

LP2985-33DBVR则是另一个低压差线性稳压(LDO),它将5V降到3.3V。根据Datasheet,在布线时,旁路电容器的放置应尽可能的接近器件的VIN和系统的GND,注意使旁路电容器连接VIN引脚和系统的GND引脚形成的环路面积最小。

绿色LED串联1KOhm保护电阻,当电路接上电源时,ON LED将发光作为提示。

\begin{figure}[htbp]
    \centering
    \includegraphics[]{Mega2560-USB-POWER.pdf}
    \caption{USB供电选择电路}
    \label{fig:Mega2560-USB-POWER}
\end{figure}

如图~\ref{fig:Mega2560-USB-POWER},LM358DR2G为增益带宽积(GBP)1MHz的通用双路运放,AO3401A则为P沟道MOS(场效应管),低电平导通。

LM358DR2G的B路运放在这里作为电压比较器使用,若VIN大于6.6V,即有锂电输入,此时运放输出高电平,PMOS关断,此时板上供电由锂电提供,同时也切断了USB,防止通过锂电给USB供电的情况出现。当VIN没有输入,即没有锂电输入,如果此时USB接通,PMOS导通,USB可以直接给板子供应5V直流。

\section{MCU附属IO}

\begin{figure}[htbp]
    \centering
    \includegraphics[]{Mega2560-Core-IO.pdf}
    \caption{MCU附属IO}
    \label{fig:Mega2560-Core-IO}
\end{figure}

如图~\ref{fig:Mega2560-Core-IO},LM358DR2G的A路运放在这里作为电压跟随器,作为黄色LED的驱动,接收PB7即D13引脚的数字输入,控制黄色LED的亮灭。

2x3的端子是ICSP (In-Circuit Serial Programming) 端口,可以通过这个6Pin端口给ATmega2560 MCU烧写程序。结合AVR-ISP (in-system programmer)比如Arduino ISP\footnote{https://store.arduino.cc/usa/arduino-isp}使用。

使用Arduino ISP,可以上传脚本并在任何基于AVR的板上烧写引导程序。通过使用外部编程器上传Sketch,可以删除引导程序bootloader。Arduino ISP还可用于刻录Arduino bootloader,因此,如果不小心损坏了bootloader,则可以用它来恢复。当使用新的ATmega MCU时,也有必要刷引导程序,将希望使用的引导程序通过USB-Serial连接上传。

SCL和SDA为I2C(Inter-Integrated Circuit)集成电路总线,是一种串行通讯汇流排,使用多主从架构,只使用两条双向漏极开路(Open Drain)(串行资料(SDA)及串行时脉(SCL))并利用电阻将电位上拉。这里两个10kOhm电阻即为I2C上拉电阻。

按下轻触开关,MCU的RESET从高电平变为低电平,MCU中的程序复位,相当于重启。

CSTCE16M0V53为16MHz晶振,内置电容,作为MCU的外部时钟晶振使用。

\section{ATmega16U2}

\begin{figure}[htbp]
    \centering
    \includegraphics[]{Mega2560-ATmega16U2-MU.pdf}
    \caption{Mega2560-ATmega16U2-MU}
    \label{fig:Mega2560-ATmega16U2-MU}
\end{figure}

如图~\ref{fig:Mega2560-ATmega16U2-MU},板上的ATmega16U2芯片充当计算机的USB端口和主处理器ATmega2560-16AU的串行端口之间的桥梁。它运行称为固件的软件(之所以这样命名,是因为一旦在芯片中对其进行编程就无法更改),该软件可以通过称为DFU(设备固件更新)的特殊USB协议\footnote{https://www.arduino.cc/en/Hacking/DFUProgramming8U2}进行更新。

\section{USB接口}

通过通用串行总线(英语:U niversal S erial B us,缩写:USB)对MCU进行程序烧写和部分数据交换,同时可以给电路板供电。

USB在速度上远比并行端口(例如EPP、LPT)与串行接口(例如RS-232)等传统电脑用标准汇流排快上许多。USB 2.0(USB 2.0 HiSpeed)为480Mbps,USB 3.0(USB 3.2 Gen1)为5Gbps,USB 3.1(USB 3.2 Gen2x1)为10Gbps,而USB 3.2(USB 3.2 Gen2x2)更达20Gbps。

由于不需要过快的数据交换速度,选用USB 2.0(USB 2.0 HiSpeed)协议进行数据交换。

% Please add the following required packages to your document preamble:
% \usepackage{booktabs}
\begin{table}[]
    \centering
    \begin{tabular}{@{}llll@{}}
    \toprule
    类别     & Type-A & Type-C & Micro-B \\ \midrule
    插头(公头) &        &        &         \\
    插座(母头) &        &        &         \\ \bottomrule
    \end{tabular}
    \caption{USB 2.0 机械电子标准一览}
    \label{tab:USB-M}
\end{table}

如表~\ref{tab:USB-M},USB的连接器分为A、B两种,分别用于主机和设备;其各自的小型化的连接器是Mini-A, Mini-B 和 Micro-A, Micro-B,另外还有Mini-AB(可支持Mini-A及Mini-B)的插口。USB 3.1版本中引入了支持正反面不区分插入的C型。

普通电脑上使用的是Type-A口,所以连接电路板时会使用Type-A转Micro USB或Type-C的USB2.0数据线。

在标准USB接口Type-A中,有四个连接器触点如表~\ref{tab:USB-4},USB信号使用分别标记为D+和D-的双绞线传输,它们各自使用半双工的差分信号并协同工作,以抵消长导线的电磁干扰。

% Please add the following required packages to your document preamble:
% \usepackage{booktabs}
\begin{table}[]
    \centering
    \begin{tabular}{@{}lll@{}}
    \toprule
    触点 & 功能(主机)             & 功能(设备)            \\ \midrule
    1  & V BUS(4.75-5.25 V) & V BUS(4.4-5.25 V) \\
    2  & D-                 & D-                \\
    3  & D+                 & D+                \\
    4  & 接地                 & 接地                \\ \bottomrule
    \end{tabular}
    \caption{标准USB Type-A连接器触点}
    \label{tab:USB-4}
\end{table}

使用Micro-USB机械电子标准的USB插座(母头),此标准常用于移动电话、平板电脑等。Micro-USB将成为移动设备数据和电源的标准接口。

Micro-USB除了第4针外,其他接口功能皆与标准USB相同如表~\ref{tab:MicroUSB}。第4针成为ID,地线在Micro-USB上连接到第5针,在Micro-USB可以悬空亦可连接到第5针。

% Please add the following required packages to your document preamble:
% \usepackage{booktabs}
\begin{table}[]
    \centering
    \begin{tabular}{@{}lll@{}}
    触点 & 功能                & 颜色 \\
    1  & V BUS(4.4–5.25 V) & 红  \\
    2  & D−                & 白  \\
    3  & D+                & 绿  \\
    4  & ID                &    \\
    5  & 接地                & 黑 
    \end{tabular}
    \caption{Micro USB连接器触点}
    \label{tab:MicroUSB}
\end{table}

也可以使用支持正反插的USB Type-C接口进行供电和烧写程序,USB Type-C接口常用于新式计算机、移动电话、平板电脑等。其通讯协议和标准见图~\ref{fig:USB-Type-C-pins}和图~\ref{fig:USB-TypeC}。

\begin{figure}[htbp]
    \centering
    \includegraphics[width=\columnwidth]{Figure-2-USB-Type-C-pins.jpg}
    \caption{USB-Type-C-pins}
    \label{fig:USB-Type-C-pins}
\end{figure}

\begin{figure}[htbp]
    \centering
    \includegraphics[width=\columnwidth]{USB-2-To-Type-C.pdf}
    \caption{USB-TypeC}
    \label{fig:USB-TypeC}
\end{figure}



\begin{figure}[htbp]
    \centering
    \includegraphics[]{Mega2560-USB.pdf}
    \caption{Mega2560-USB}
    \label{fig:Mega2560-USB}
\end{figure}

如图~\ref{fig:Mega2560-USB},

USB口外壳和内部引线之间并联了可变电阻和磁珠

\section{扩展接口}

三明治扩展。

\section{接口}
