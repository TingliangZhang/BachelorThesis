\thusetup{
  %******************************
  % 注意:
  %   1. 配置里面不要出现空行
  %   2. 不需要的配置信息可以删除
  %******************************
  %
  %=====
  % 秘级
  %=====
  secretlevel={秘密},
  secretyear={10},
  %
  %=========
  % 中文信息
  %=========
  ctitle={分布式能源资源的自治优化仿真平台研发},
  cdegree={工学学士},
  cdepartment={电机工程与应用电子技术系},
  cmajor={电气工程及其自动化},
  cauthor={张庭梁},
  csupervisor={钟海旺副教授},
  % cassosupervisor={陈文光教授}, % 副指导老师
  % ccosupervisor={某某某教授}, % 联合指导老师
  % 日期自动使用当前时间,若需指定按如下方式修改:
  % cdate={超新星纪元},
  %
  % 博士后专有部分
  % catalognumber     = {分类号},  % 可以留空
  % udc               = {UDC},  % 可以留空
  % id                = {编号},  % 可以留空: id={},
  % cfirstdiscipline  = {计算机科学与技术},  % 流动站(一级学科)名称
  % cseconddiscipline = {系统结构},  % 专 业(二级学科)名称
  % postdoctordate    = {2009 年 7 月——2011 年 7 月},  % 工作完成日期
  % postdocstartdate  = {2009 年 7 月 1 日},  % 研究工作起始时间
  % postdocenddate    = {2011 年 7 月 1 日},  % 研究工作期满时间
  %
  %=========
  % 英文信息
  %=========
  etitle={Research and development of autonomous optimization simulation platform for distributed energy resources},
  % 这块比较复杂,需要分情况讨论:
  % 1. 学术型硕士
  %    edegree:必须为Master of Arts或Master of Science(注意大小写)
  %             “哲学、文学、历史学、法学、教育学、艺术学门类,公共管理学科
  %              填写Master of Arts,其它填写Master of Science”
  %    emajor:“获得一级学科授权的学科填写一级学科名称,其它填写二级学科名称”
  % 2. 专业型硕士
  %    edegree:“填写专业学位英文名称全称”
  %    emajor:“工程硕士填写工程领域,其它专业学位不填写此项”
  % 3. 学术型博士
  %    edegree:Doctor of Philosophy(注意大小写)
  %    emajor:“获得一级学科授权的学科填写一级学科名称,其它填写二级学科名称”
  % 4. 专业型博士
  %    edegree:“填写专业学位英文名称全称”
  %    emajor:不填写此项
  edegree={Bachelor of Engineering},
  emajor={Electric Engineering},
  eauthor={Zhang Tingliang},
  esupervisor={Associate Professor Zhong Haiwang},
  %eassosupervisor={Chen Wenguang},
  % 日期自动生成,若需指定按如下方式修改:
  % edate={December, 2005},
  %
  % 关键词用“英文逗号”分割
  ckeywords={共识算法, 智能电网, 分布式算法, 可视化硬件平台, 电力市场},
  ekeywords={Consensus Algorithm, Smart Grid, Distributed Algorithm, Visual Hardware Platform, Electricity Market}
}

% 定义中英文摘要和关键字
\begin{cabstract}

  本文介绍了用于一个智能电网算法评估的可视化测试平台,同时也是用于开发桌面实体集群交互界面的可扩展软硬件开源平台。

  该平台包括一组定制设计的3全向轮机器人(每个直径10厘米),通过覆盖在活动表顶部的微点图案进行的高精度定位,以及用于应用程序开发和控制的软件框架,同时仍保持价格合理 (在原型机阶段,每个价格约为30美元)。 我们通过使用该平台开发新的简化的智能电网分布式算法应用来说明桌面集群用户界面的潜力。

  基于区块链技术和共识算法,面向泛在电力物联网建设,考虑实际应用环境中的问题(如噪音、干扰、局部通信中断等问题),开发了一套测试平台并模拟实际应用,以应对传统集中式优化通信负担过重、过度依赖部分节点、灵活性不足等问题的挑战。

  通过嵌入式可视化集群硬件平台来直观的表现算法的运行,同时也探索一种新的展示算法的方式。

  本文的创新点主要有:

  \begin{itemize}
    \item 将分布式算法用于电力市场调度中,实现了去中心化,增强了电力系统的可靠性。
    \item 开发了一套集群可视化平台,探索了一种新的展示算法的方式。
    \item 考虑了硬件存在的通信中断和恶意节点等现实问题。
  \end{itemize}


\end{cabstract}

% 如果习惯关键字跟在摘要文字后面,可以用直接命令来设置,如下:
% \ckeywords{\TeX, \LaTeX, CJK, 模板, 论文}

\begin{eabstract}
  In this article, we present a visualized swarm testbed for smart grid algorithm evaluation, also an extendable open-source open-hardware platform for developing tabletop tangible swarm interfaces.

  The platform consists of a collection of custom-designed 3 omni-directional wheels robots each 10 cm in diameter, high accuracy localization through a microdot pattern overlaid on top of the activity sheets, and a software framework for application development and control, while remaining affordable (per unit cost about 30 USD at the prototype stage). We illustrate the potential of tabletop swarm user interfaces through a set of smart grid algorithm application scenarios developed with the platform.

  Based on blockchain technology and consensus algorithms, for the construction of ubiquitous electric power IoT, considering the problems in the actual application environment (such as noise, interference, local communication interruption, etc.), a test platform has been developed and simulated for practical applications Traditional centralized optimization challenges such as excessive communication burden, excessive dependence on some nodes, and insufficient flexibility.

  The embedded visualization cluster hardware platform is used to intuitively express the operation of the algorithm, and also explore a new way to display the algorithm.
  
  The innovations of this article are:

  \begin{itemize}
    \item The distributed algorithm is used in power market dispatching to achieve decentralization and enhance the reliability of power systems.
    \item developed a cluster visualization platform and explored a new way to display algorithms.
    \item takes into account the practical problems of communication interruptions and malicious nodes in the hardware.
  \end{itemize}


\end{eabstract}

% \ekeywords{\TeX, \LaTeX, CJK, template, thesis}
