\chapter{相关代码}
\label{cha:Code}

\section{平均值分布式算法收敛性检测}
\label{sec:Consensus}
% minted需要latexmk 加入 --shell-escape 参数,即 $ latexmk -xelatex --shell-escape main.tex
% \usepackage[cache=false]{minted}  % 代码高亮

% 如果出现一堆 Undefined control sequence.  那应该是没加 [cache=false]

% Package minted Error: You must have `pygmentize' installed to use this packag
% Solution:应该是Python的包Pygments的问题了,用pip安装了这个包
%    pip install pygments

% Windows 10 1903下Python命令关联到了Microsoft Store里面,如果没安装python:
% 去 https://www.python.org/downloads/ 下载下来 直接默认路径+添加到PATH即可

% 你可以通过以下命令来判断是否已安装pip:
% pip --version
% 下载pip:https://pypi.python.org/pypi/pip#downloads
% 之后 python setup.py install 安装
% 搜索 环境变量 
% 把C:\Users\ZTL\AppData\Local\Programs\Python\Python37-32\Scripts 加到Path里

% 必要时注意重启VS Code以更新环境变量


% \begin{minted}{c++}
%     int main() {
%         printf("hello, world");
%         return 0;
%     }
% \end{minted}


% Using different styles
% The full syntax is \usemintedstyle[hlanguagei]{hstylei}
% To get a list of all available stylesheets, see the online demo at the Pygments website or execute the following command on the command line:
% $ pygmentize -L styles

% Supported languages
% Pygments supports over 300 different programming languages, template languages, and other markup languages. To see an exhaustive list of the currently supported languages, use the command
% $ pygmentize -L lexers

% Finally, there’s the \inputminted command to read and format whole files. Its syntax is \inputminted[hoptionsi]{hlanguagei}{hfilenamei}.
\inputminted[mathescape, linenos, breaklines]{python3}{Code/Consensus.py}
% mathescape  在注释中显示Latex格式的公式
% linenos  显示行号
% breaklines  自动换行


\section{Jupyter Lab UI 界面}
\label{sec:JupyterLabUI-Code}
\inputminted[mathescape, linenos, breaklines]{python3}{Code/UI.py}

\section{XBee发送数据}

\subsection{NetworkModificationsSample}
\label{sec:NetworkModificationsSample}
\inputminted[mathescape, linenos, breaklines]{python3}{Code/NetworkModificationsSample/NetworkModificationsSample.py}

\subsection{SendDataSample}
\label{sec:SendDataSample}
\inputminted[mathescape, linenos, breaklines]{python3}{Code/SendDataSample/SendDataSample.py}

\subsection{SendDataAsyncSample}
\label{sec:SendDataAsyncSample}
\inputminted[mathescape, linenos, breaklines]{python3}{Code/SendDataAsyncSample/SendDataAsyncSample.py}

\subsection{SendBroadcastDataSample}
\label{sec:SendBroadcastDataSample}
\inputminted[mathescape, linenos, breaklines]{python3}{Code/SendBroadcastDataSample/SendBroadcastDataSample.py}

\section{实物可视化界面DEMO源码}
\label{sec:MisakaCarV1}
\inputminted[mathescape, linenos, breaklines]{c}{Code/MisakaCarV1/MisakaCarV1.ino}

% 以下为非必须

\section{DRV8825驱动微型步进电机加速及定距移动测试}

\subsection{3步进电机加速测试}
\label{sec:Stepper-3}
\inputminted[mathescape, linenos, breaklines]{c}{Code/Stepper-3/Stepper-3.ino}

\subsection{坐标变换及运动函数封装}
\label{sec:Stepper-3-v2}
\inputminted[mathescape, linenos, breaklines]{c}{Code/Stepper-3-v2/Stepper-3-v2.ino}


\section{TB6612FNG}
\label{sec:TB6612FNG-Functionality-Test}
\inputminted[mathescape, linenos, breaklines]{c}{Code/TB6612FNG-Functionality-Test/TB6612FNG-Functionality-Test.ino}

\section{Arduino as ISP}
\label{sec:ArduinoISP}
\inputminted[mathescape, linenos, breaklines]{c}{Code/ArduinoISP/ArduinoISP.ino}

\section{WS2812B}
\label{sec:WS2812B-Functionality-Test}
\inputminted[mathescape, linenos, breaklines]{c}{Code/WS2812B-Functionality-Test/WS2812B-Functionality-Test.ino}


\section{XBee通信 RX and TX}

\subsection{接收Frame数据包}
\label{sec:Series2_Rx}
\inputminted[mathescape, linenos, breaklines]{c}{Code/Series2_Rx/Series2_Rx.ino}

\subsection{接收Frame数据包并通过软件串口发送}
\label{sec:Series2_Rx_Nss}
\inputminted[mathescape, linenos, breaklines]{c}{Code/Series2_Rx_Nss/Series2_Rx_Nss.ino}

\subsection{发送Frame数据包}
\label{sec:Series2_Tx}
\inputminted[mathescape, linenos, breaklines]{c}{Code/Series2_Tx/Series2_Tx.ino}